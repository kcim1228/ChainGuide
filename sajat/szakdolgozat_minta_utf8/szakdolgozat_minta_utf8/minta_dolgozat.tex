% BEÁLLÍTÁSOK - JOBB NEM VÁLTOZTATNI
\documentclass[final]{ubb_dolgozat}
\usepackage{definitions}
\sloppy
\frenchspacing
%%



% milyen nyelveken akarunk forráskódot megjeleníteni
\lstloadlanguages{Clean,Prolog,Matlab,C,C++}

% ezt be lehet tenni MINDEGYIK megjelenítendő kód elé.
\lstset{language=Matlab}


%%%%%%%%%%%%%%%%%%%%%%%%%%%%%%%%%%%%%%%%%%%%%%%
%%!!          EZT KELL VÁLTOZTATNI       !!%%%%
%%     A DOLGOZAT CÍMOLDALÁNAK ELEMEI        %%

%% MELYIK ÉVBEN ADJUK LE
\submityear{%
2015
}
%% MELYIK HÓNAPBAN ADJUK LE
\submitmonthHU{%
Július
}
\submitmonthRO{%
Iulie
}
\submitmonthEN{%
July
}

\titleHU{%
Chain-Guide, a biciklibarát webalkalmazás
}

% Amennyiben szükséges, az alábbi sorokat ki kell komment-ezni és 
% beírni a megfelelő címeket

\titleEN{%
Chain-Guide, the cyclists webapplication
}

\titleRO{%
Chain-Guide
}

\author{%
Kátay Csilla
}

%%
\tutorHU{%
dr. Ruff Laura, egyetemi adjunktus\\
%{\large Babe\c{s}--Bolyai Tudományegyetem,\\
% Matematika és Informatika Kar}% ha különbözik, akkor fel kell tűntetni
}
%%
\tutorRO{%
Lector dr.  Ruff Laura\\ % {\large Universitatea Babe\c{s}--Bolyai,\\ % dacã diferã!!!
% Facultatea de Matematic\u{a} \c{s}i Informatic\u{a} }%
}
%%
\tutorEN{%
 Ruff Laura, assistant professor
% {\large Babe\c{s}--Bolyai University,\\
% Faculty of Mathematics and Informatics}
}

%\includeonly{bevezet}


\begin{document}

%% ABSTRAKT
\begin{abstractEN} % ANGOL VÁLTOZAT

% a lenti részt értelemszerűen ki kell tölteni a dolgozat angol kivonatával.
% A BEGIN ... END között CSAK A SAJÁT SZÖVEG kell, hogy legyen.
% Az utolsó mondatot benne kell hagyni, mely által kijelentitek, hogy a munkátok SAJÁT.


{ 
	

	
  
  \center{
  	 The topic of dissertation is the self-developed, Java-based web application called Chain-Guide. The app is designed to help cyclists to find suitable routes, services(shop, rental, service) for them.
}
	  \center{ As part of the application , users can plan cycle-friendly routes avoiding the busy streets of the city and can discover new hiking trails as well. Besides, there is possibility to choose from a variety of services based on different searching criterias, and also to evaluate them. The maintainability of the application is provided by an administrative interface. The information about cycling routes is served by the OpenCycleMap through MapQuest API. The web-based user interface uses the Vaadin framework, while the communication with the database was based on the Hibernate framework.
}
	  \center{ Its strength lies on the fact that is unique in the market among the cyclist web applications wich are available in our country and provide additional information as well.}

  
  
	
	\vfill


}

This work is the result of my own activity. I have neither given nor received unauthorized assistance on this work.

\end{abstractEN}

% ez a címoldal része
\maketitle

%% 

% a dolgozat tartalomjegyzéke -- ez automatikusan generálódik a STRUKTÚRA alapján.
{ \baselineskip 1ex
  \parskip 1ex
  \tableofcontents
}


%%%%%%%%%%%%%%%%%%%%%%%%%%%%%%%%%%%%%%%%%%%%%%%%%%%%%%%%%%%
%%%%%%%%%%         a dolgozat tartalma         %%%%%%%%%%%%

% ajánlott külön file-okba írni az egyes fejezeteket,
% ugyanis úgy jobban át lehet látni.


% a bevezetõ fejezet FILE-ja.
%!TEX root = minta_dolgozat.tex
%%%%%%%%%%%%%%%%%%%%%%%%%%%%%%%%%%%%%%%%%%%%%%%%%%%%%%%%%%%%%%%%%%%%%%%
\chapter{Bevezető}\label{ch:BEVEZET}
%%%%%%%%%%%%%%%%%%%%%%%%%%%%%%%%%%%%%%%%%%%%%%%%%%%%%%%%%%%%%%%%%%%%%%%


\setlength{\parindent}{4em}
\setlength{\parskip}{1em}

 \par	A dolgozat témája a saját fejlesztésű,  Chain-Guide nevű,   \textit{Java} alapú webalkalmazás megvalósítása. Az alkalmazás célja, hogy segítse a biciklisek közlekedését, azon városokban is, ahol a  \textit{Google Maps} ezen része  (cycling direction and bike %megj.: pl. \href{http://www.typotex.hu/latex.htm}{TYPOTEX kiadó}
  routes\footnote{Google cycling direction and bike roots: \\ \href{http://googlepolicyeurope.blogspot.ro/2013/05/bringing-biking-directions-to-more-of.html}{http://googlepolicyeurope.blogspot.ro/2013/05/bringing-biking-directions-to-more-of.html}}) még nem elérhető, pedig a kerékpár utak száma növekvőben van. Az alkalmazás az útvonalválasztás mellett, a bicikli orientált szolgáltatások (kölcsönzés, szervízelés stb.) terén is segítséget nyújt felhasználóinak. 
 \par	Az alkalmazás keretén belül, a felhasználók kerékpárbarát útvonalakat tervezhetnek, elkerülve ezáltal a város forgalmas utcáit, vagy  új túraösvényeket, esetleg extrém parkokat is felfedezhetnek a kalandra vágyók. A szolgáltatások terén sem marad alul, hiszen lehetőséget nyújt, hogy a felhasználó megtalálja a neki megfelelő üzletet, szervízt vagy kölcsönzőt. A keresési feltételek listájában mind a közvélemény, mind a nyitvatartás és közelség is helyet foglal. Mindemellett véleményezésre is lehetőségük nyílik a felhasználóknak. 
 \par	Az alkalmazás  karbantarthatóságát egy adminisztrációs felület biztosítja, mely elengedhetetlen ahhoz, hogy ez naprakész információkat használjon a különböző szolgáltatásokhoz. A biciklizés szempontjából fontosabb út-információkat az  \textit{OpenCycleMap} \cite{OpenCycleMap} (a legelterjedtebb biciklis réteggel rendelkező térkép-szolgáltatás mely világszerte elérhető, beleértve Romániát is és szabadon aktualizálható, kiegészíthető) adja, melyet a  \textit{MapQuest API}-n \cite{MapQuestJsApi} keresztül ér el az alkalmazás. A webes felületet a  \textit{Vaadin} \cite{Vaadin}  keretrendszer segítségével valósítottuk meg, míg az adatbázissal történő komunikáció a  \textit{Hibernate} \cite{Hibernate} programkönyvtár segítségével lett kivitelezve. 
 \par	\textcolor{red}{A dolgozat szerkezetét illetően .... főbb részre bontható.......
		(itt akkor ezt utólag fogom hozzáírni)}
\par .
\par.
\par .





 \par	Erőssége abban rejlik, hogy egyedi a piacon kínálkozó biciklis-alkalmazások közt, amelyek hazánkban, környékünkön is elérhetőek. A szolgáltatásokkal járó extra információk is az alkalmazás előnyeiként említhetőek meg, ugyanúgy mint a felhasználóbarát megjelenítés vagy az egyszerű használat. 




%!TEX root = minta_dolgozat.tex




\setlength{\parindent}{4em}
\setlength{\parskip}{1em}
\chapter{Felhasznált technológiák}\label{ch:FELH}

\begin{osszefoglal}
	Ebben a fejezetben a felhasznált technológiák kerülnek a középpontba. Bővebben, a \textit{Vaadin}\cite{Vaadin} és \textit{Hibernate}\cite{Hibernate} keretrendszerekről,  a \textit{MapQuest}\cite{MapQuestJsApi} és \textit{OpenCycleMap}\cite{OpenCycleMap} API-król illetve az \textit{Abstract Factory} tervezési mintáról szól a dolgozat ezen része.
\end{osszefoglal}

%%%%%%%%%%%%%%%%%%%%%%%%%%%%%%%%%%%%%%%%%%%%%%%%%%%%%%%%%%%%%%%%%%%%%%%
\section{Vaadin}\label{sec:FELH:va}

\par Az alkalmazás webes felületét a \textit{Vaadin} keretrendszer segítségével valósítottuk meg. Ez egy nyílt forráskódú webalkalmazás-keretrendszer, amellyel interaktív web tartalom készíthető \textit{Java} nyelven és a hagyományos \textit{GUI} (grafikus felhasználói felület) fejlesztéshez hasonlítható. 
\par A \textit{Vaadin} megjelenítésre a \textit{Google Web Toolkit}-et (AJAX fejlesztői eszköztár) használja, míg a szerver oldal alapját a \textit{Java Servlet}  (Java objektum, mely \textit{HTTP}\footnote{Hypertext Transfer Protocol} kérést dolgoz fel és \textit{HTTP} választ generál) technológia képezi. Maga a kódolás \textit{Java} nyelven történik, a \textit{GWT} (Google Web Toolkit) ezt \textit{Javascript} forráskódra alakítja át, ami a böngészőben kerül majd futtatásra. A \textit{GWT} csupán egy vékony megjelenítési réteg a Vaadin esetében, mivel az alkalmazás logika a szerver oldalon helyezkedik el teljes mértékben. A kommunikációra \textit{AJAX} (Asynchronous JavaScript and XML) technológiát használ, míg az adatok \textit{JSON} (Javascript Object Notation) szabvány szerint vannak kódolva.
\par A\textit{ Vaadin} előnyére szolgál az, hogy komponensei kiegészíthetők \textit{GWT}-Widgetekkel illetve lehetőség van  \textit{CSS}-el (Cascading Style Sheets) való formázásra is, ami elengedhetetlen egy felhasználóbarát felület megalkotásában. Mivel az \textit{Eclipse} rendelkezik megfelelő beépített \textit{Vaadin} modulokkal, ez is a fejlesztés javára szolgált. A \textit{GWT} fordítónak köszönhetően a legtöbb modern böngészővel kompatibilis. Mivel ezen keretrendszer esetében szükség esetén összekapcsolható a \textit{Java} a\textit{ Javascript} kóddal, a \textit{Vaadin} ebből a szemszögből is egy helyes választásnak bizonyult. A Chain-Guide alkalmazás keretén belül így könnyebben hozzá tudtunk férni a térképszolgáltató függvényeihez a \textit{MapQuest Javascipt API}-ján keresztül. 
\par A \textit{Vaadin 7} és az annál újabb verziók  (a megvalósított webalkalmazás a 7.2.4-es verziót használja) a következő webböngészőkkel kompatibilisek: \footnote{Vaadin Features: \\ \href{https://vaadin.com/features}{https://vaadin.com/features}}
\par
\begin{itemize}
  \item Google Chrome 23 vagy újabb
  \item Internet Explorer 8 vagy újabb
  \item Mozilla Firefox 17 vagy újabb
  \item Opera 15 vagy újabb
  \item Safari 6 vagy újabb 
\end{itemize}

\parEgy Vaadin alkalmazás felépítése és működése:

\parA webes modul megalkotásánál egy új Vaadin projektet hoztunk létre, az Eclipse Java IDE\footnote{ nyílt forráskódú platformfüggetlen szoftverkeretrendszer } fejlesztői környezetben, a Vaadin modul letöltésének segítségével. A projekt nevének megadása után lehetőség van konfigurálni a szükséges technológiákat, mint például a webszerver megadása vagy a Vaadin verziószámának a beállítása. Ezen műveletek végrehajtása után a Vaadin kigenerál egy kezdetleges projekt szerkezetet egy UI osztállyal, illetve a telepítésleíró (deployment descriptor) web.xml állománnyal. 

\par A Vaadin két fejlesztési modellt biztosít alkalmazásai számára: a kliens és szerver oldali modelleket. Az alkalmazás felépítését illetően a szerver oldali architektúrára kerül a hangsúly, mivel fejlesztés kizárólag szerver oldalon történik. A kliens oldal, a Vaadin Kliens-Oldali Motor (Vaadin Client-Side Engine) segítségével, mely AJAX\footnote{ Asynchronous JavaScript And XML } technológiára épül, csak megjeleníti a felhasználói felületet (UI-t) a böngészőben. Emelett, a kliens oldal a különböző widgetek használatát, fejlesztését is támogatja (Java környezetben), amik Javascriptre fordítodnak és úgy hajtódnak végre a böngészőben. A két modul (kliens és szerver) közti kommunikáció egyszerűen megvalósítható mindkét irányba.

\par Egy Vaadin alkalmazás futás közbeni architektúrája a ... ábrán látható, amely egy egyszerű illusztráció a kliens és szerver oldalak közti kapcsolatról, abban a pillatnatban amikor a kliens oldali kód már be van töltődve a böngészőbe. A ábra alapján egy Vaadin alkalmazás tartalmaz: egy kliens-oldali API-t, egy szerver-oldali API-t, különböző widgeteket és témákat mindkét oldalon (felhasználóbarát megjelenítés), illetve egy adat modellt, amely lehetővé teszi, hogy a szerver oldali komponensek közvetlenül tudjanak kapcsolódni az adatainkhoz. A kliens oldal tartalmaz még egy Vaadin Fordítot (Vaadin Compiler), amely segítségével a Java kódót Javascriptre fordítja a keretrendszer. A szerver oldali Vaadin alkalmazás servlet-ként (általában VaadinServlet) fut egy Java web szerveren, kiszolgálva a HTTP kéréseket. A Vaadin Servlet osztály fogadja a kliens kéréseit és nekik megfelelő eseményekként továbbítja őket az alkalmazásban megadott esemény-figyelőkhöz (listener) a UI-on belül. A Kliens-Oldali Motor (engine), mely a böngészőben fut, fogadja a kéréseket és végrehajta a kért változtatásokat a weboldalon.
\par A felhasználói felület (User Interface): A GUI felépítéséhez, minden Vaadin alkalmazás esetében szükség van a com.vaadin.ui.UI absztrakt osztály kiterjesztésére, ez képviseli az alkalmazás belépési pontját amikor az alkalmazás url-jét beírják aböngészőbe.
\par A felhasználói felület komponensei: A UI komponensekből épül fel. Minden szerver-oldali komponensnek van egy kliens oldali megfelelője, amelyik közvetlen kapcsolatban áll a felhasználóval. A felhasználók által kiváltott események, a szerver oldali komponenseken keresztül továbbítódnak az alkalmazás back-end részéhez.
\par Kliens-Oldali-Motor: Feladata, hogy a felhasználói felületet megjelenítse a böngészőben (kliens-oldali widgetek segítségével), vagyis fogadja a kéréseket (HTTP vagy HTTPS) és végrehajtja a nekik megfelelő változásokat a weboldalon.
\par Vaadin Servlet: A Vaadin Servlet osztály kéréseket fogad különböző kliensektől, eldönti, hogy melyik felhasználói munkamenethez (user session) tartoznak (cookiek segítségével), és a nekik megfelelő munkamenetekhez (session) továbbítja őket.
\par Témák: A Vaadin különválasztja az elemek szerkezetét és megjelenítését a felhasználói felületen. Ez utóbbira CSS-t vagy Sass-t használ.



%
\section{Hibernate}\label{sec:FELH:hi}

\par Az alkalmazás backend része a \textit{Hibernate} programkönyvtár segítségével lett kivitelezve. A \textit{Hibernate} egy \textit{ORM} (objektum-relációs leképezést megvalósító) keretrendszer \textit{Java} platformra, melynek legfőbb célja az adatbázissal történő kommunikáció leegyszerűsítése. Segítségével az adatbázisban lévő rekordokat objektumként kezelhetjük és állapotmegörző módon adattáblákban tárolhatjuk. Legfőbb jellemzője ezek mellett, hogy adatbázis függetlenséget biztosít.
\par A \textit{HQL} (Hibernate Query Language) a \textit{Hibernate} saját adatlekérdező nyelve, mely lehetőséget teremt lekérdezések írására és futtatására (\textit{SQL} tudás nélkül). A keretrendszer ezen \textit{HQL} lekérdezésekből generálja az adatbáziskezelő rendszer számára megfelelő \textit{SQL} (Structured Query Language) lekérdezéseket. Így, a fejlesztők előnyére, megkíméli őket az eredményhalmazok objektumokra történő konverziójától.
\par Az adattáblák és osztályok közti leképezéseket vagy mappinget \textit{XML} (Extensible Markup Language), esetleg \textit{Java} annotációk segítségével valósítja meg. 


\lstset{language=XML}
\begin{lstlisting}

<hibernate-mapping>
    <class name="edu.ubbcluj.backend.model.Rating" table="rating" catalog="bike" optimistic-lock="version">
        <id name="id" type="int">
            <column name="id" />
            <generator class="identity" />
        </id>
        <many-to-one name="services" class="edu.ubbcluj.backend.model.Services" fetch="select">
            <column name="serviceId" />
        </many-to-one>
        <many-to-one name="users" class="edu.ubbcluj.backend.model.Users" fetch="select">
            <column name="userId" />
        </many-to-one>
        <property name="rate" type="java.lang.Integer">
            <column name="rate" />
        </property> }
    </class>
</hibernate-mapping>
\end{lstlisting}

\par 
Az fenti példában a Chain-Guide alkalmazás Rating.hbm (Hibernate Mapping File) állomány tartalma látható. Ez az XML file az Értékelés (Rating) adattábla és a neki megfelelő modell osztály között teremti meg a kapcsolatot. A {\tt <generator class="identity" /> } tag az egyedi azonosító generálására szolgál, amely az {\tt id} nevezetű, elsődleges kulcs típusú adattagot jellemzi. A {\tt many-to-one} tag név az egy-a-többhöz kapcsolat leírására szolgál míg a {\tt  property} név alatt az olyan tábla adattagokat adjuk meg, melyek nem állnak kapcsolatban más táblák mezőivel.

\par A beépített ,,dirty check”\footnote{\textit{Hibernate} jellemzője, a keretrendszer leellenőrzi, hogy egy adott objektumon történt-e változás vagy sem, és ha igen, csak akkor hajtja vérgre a frissítést (update)} is pozitívumként emelhető ki, hiszen megakadályozza a felesleges beszúrásokat az adatbázisba. A \textit{Hibernate} esetében két féle betöltési módról beszélhetünk: lusta betöltés (lazy loading) és mohó betöltés (lazy = false). Lusta betöltés esetén csak akkor fut le a  lekérdezés, amikor először hivatkozunk az objektumra, míg a mohó esetén az már az objektum betöltésekor. Átlátható módon biztosítja a \textit{Plain Old Java Object}-ek (POJO) perzisztenciáját a felhasználók számára (az egyetlen követelmény, hogy az osztálynak legyen egy argumentum nélküli konstruktora).
%
\section{Apache Maven}\label{sec:FELH:am}

A projekt moduljainak egyszerű menedzselését a \textit{Maven} szoftver biztosította, melynek legfőbb célja az összeállítási (build)  folyamatok automatizálása. Előnyére szolgál, hogy dinamikusan is le tud tölteni komponenseket, szoftver-csomagokat, ha szükséges. Egy \textit{XML} file (POM) segítségével adhatjuk meg, hogyan legyen a projekt felépítve, milyen sorrendben legyenek buildelve a különböző modulok, illetve, hogy milyen külső függőségeket, pluginokat, komponenseket használjon. A buildelés szabványosítása által a tervezési minták terjesztése a célja.
\par Az alábbi példában egy részlet látható az alkalmazás backend\footnote{adat elérési réteg} részének a  pom.xml állományából. A részletben függőségként a \textit{Hibernate} és \textit{MySQL} csatlakozók (connector) láthatóak, amiket az adatbázissal történő kommunikációra használ a rendszer.
\textcolor{red}{ megsyamozni a kepeket!!!!!!!!!!!!!! es az ALABBI PELDA szvakahoz tenni hivatkozast}

\lstset{language=XML}
\begin{lstlisting}
<dependencies>
  	<dependency>
		<groupId>org.hibernate</groupId>
		<artifactId>hibernate-core</artifactId>
		<version>4.3.8.Final</version>
	</dependency>
		<dependency>
			<groupId>mysql</groupId>
			<artifactId>mysql-connector-java</artifactId>
			<version>5.1.34</version>
		</dependency>
  </dependencies>
\end{lstlisting}

\par Az összeállítási (build) folyamat automatizálására az alábbi példa emelhető ki. A példában a {\tt <build> tag}-ek közé a \textit{CSS} állományok automatikus lefordítását és frissítését kérjük a rendszertől a projekt build-elésével együtt.
\lstset{language=XML}
\begin{lstlisting}
	<build>
		<finalName>bike-web</finalName>
		<plugins>
			<plugin>
                <groupId>com.vaadin</groupId>
                <artifactId>vaadin-maven-plugin</artifactId>
                <version>7.2.4</version>
                <executions>
                    <execution>
                        <goals>
                            <goal>clean</goal>
                            <goal>resources</goal>
                            <goal>update-theme</goal>
                            <goal>compile-theme</goal>
                        </goals>
                    </execution>
                </executions>
            </plugin>
		</plugins>
	</build>
\end{lstlisting}


%
\section{MapQuest}\label{sec:FELH:mq}

Az alkalmazás esetében a térképpel kapcsolatos információkat és függvényeket a \textit{MapQuest} szolgáltatta. Ez egy amerikai ingyenes online térkép szolgáltatás, mely hazánkban is elérhető, és a webes desktop és mobil alkalmazásokat is egyaránt támogatja. A különböző API-k és szolgáltatásai révén egyszerűen integrálható. A fejlesztők számára szükséges egy \textit{AppKey} (Aplication Key), egy egyedi kulcs, mely által a \textit{MapQuest} szerverei azonosítani tudják az alkalmazásunkat, annak érdekében, hogy helyes válaszokat térítsenek vissza kéréseinkre. Ez ingyenesen igényelhető regisztráció\footnote{Application Keys (AppKeys): \\ \href{http://developer.mapquest.com/fr/web/info/account/app-keys}{http://developer.mapquest.com/fr/web/info/account/app-keys}}  által. A \textit{MapQuest}  \mbox{út-,} közlekedés- és forgalommal kapcsolatos információit alapértelmezetten az \textit{OpenStreetMap} (szabadon szerkeszthető és felhasználható térkép) szolgáltatja. E térkép kerékpár rétege az \textit{OpenCycleMap}, amely biciklis szempontból hasznos informácókat szolgáltat világszerte, beleértve Romániát is. 

\par A {\tt javacript} állományok  integrációjáról (\textit{Vaadinba} való beágyazásáról) a \textcolor{red}{referencia oda!!} részben található egy részletesebb leírás.  Röviden összefoglalva a  \textit{MapQuestJavascript API}-t nem szükséges letölteni mint különálló {\tt javascript} állomány, csupán az elérési útvonalat kell megadni  {\tt javasciptes annotáció } segítségével (az {\tt AppKey} -el együtt), hasonlóan a saját {\tt javascipt} állományok betöltéséhez. 
\lstset{language=Java}
\begin{lstlisting}
import com.vaadin.annotations.JavaScript;

@JavaScript({"http://open.mapquestapi.com/sdk/js/v7.2.s/mqa.toolkit.js?key=APPKEY","mylib.js"})

\end{lstlisting}
\textcolor{red}{ megszamozni es hivatkozni ra}

\par A \textit{Javascript Maps API} egyike a legelterjedtebb \textit{MapQuest}-es szolgáltatásoknak. Funkcionalitásait illetően lehetőséget nyújt térképes felületek létrehozásához különböző extra opciókkal (live traffic, self-localization stb.), vannak beépített útkereső függvényei, melyeknek paraméterként a {\tt bicycle} kulcsszót megadva biciklibarát útvonalak rajzoltathatóak ki a térképre. A {\tt geocoding} modul átjárhatóságot biztosít a koordinátákat tartalmazó {\tt LatLng} objektumok és a direkt módon megadott címek közt, melyek ugyanazt a pontot határozzák meg a térképen. A különböző eseménykezelő függvényeivel interaktívabbá varázsolhatók az alapműveletek, illetve a térkép objektumok is felülírhatók, személyre szabhatóak, egy felhasználóbarát felület kialakításának érdekében.  
\par 
Az alkalmazás legtöbbet használt moduljaként a {\tt Geocoding} modul emelhető ki. Konkrétabban a {\tt geocodeAndAddLocation} és a {\tt reverseGeocodeAndAddLocation} függvények hangsúlyozhatók ki, melyek segítségével a {\tt LatLng} objetumokból valós címek nyerhetők és jeleníthetők meg a térképen illetve fordítva. Ezek keretén belül a {\tt POI}\footnote{a térképen egy pontot megjelenítő objektum} objektumok is személyre szabhatóak, változtatható az ikonjuk, info-ablakuk, illetve felülirhatók a rájuk értelmezett események is (kattintás, mozgatás stb.). Az alkalmazás elengedhetetlen része a minden oldalon megjelenő térkép objektum amelyet az {\tt MQA} modul {\tt TileMap} függvényének meghívásával rajzolhatunk ki a paraméterként megadott opciókkal. Ezen paraméter egy olyan adatszerkezet, melyben megadható, hogy hova töltődjön be a térkép, mi legyen a középpontja, mekkora legyen az alapértelmezett közelítés, stb. A  {\tt Routing} modul is az alkalmazás alapját képezi, hiszen elengedhetetlen két pont közötti útvonal megjelenítéséhez. Az  {\tt AddRoute} függvénynek megadhatóak  {\tt LatLng} objektumok és címek is egyaránt. Az  {\tt options} struktúrában meghatározható a keresés típusa, például  {\tt bicycle}, amely egy olyan útvonalat jelenít meg A és B pontok között mely a legbiciklibarátabbnak nevezhető (bicikliutak, forgalom mentes utcák, kerékpárbarát környezet ). A  {\tt shortest} opcióval a fizikai értelemben vett legrövidebb útszakaszt kapjuk válaszként. Az útkeresés típusa mellett megadhatók még a megjelnítésre vonatkozó extra opciók is, amellyel rövid útmutatót is megjeleníthetünk az adott útvonalra.

\textcolor{red}{ ( majd utólag beírom a hivatkozást a gyakorlati rész megfelelő fejezetére, ahol a konkrét példák lesznek).}
%
\section{OpenCycleMap}\label{sec:FELH:ocm}

Az \textit{OpenCycleMap} az \textit{OpenStreetMap} térképes szolgáltatás egy rétege (layer). Az \textit{OpenStreetMap} (OSM) egy szabadon szerkeszthető és felhasználható térképfejlesztés. A térképek egyszerű helyismeretből vagy hordozható {\tt GPS } eszközökből, légifotókból származó adatokra épülnek, amelyeket az {\tt Open Database License} (nyílt adatbázis) tárol. A regisztrált felhasználóknak lehetőségük van szerkeszteni a vektor alapú adatokat illetve {\tt GPS} nyomvonalakat is feltölthetnek. A romániai adatok szempontjából a legjobban aktualizált térképszolgáltató, illetve biciklis információk terén a legjelentősebb.
\par 	A biciklis réteg tartalmazza az összes nemzetközileg elismert bicikliutat illetve a lokális és regionális kerékpár utak javát, ezen kívül megjeleníthetők túraútvonalak, bicikli üzletek és parkolók is egyaránt, ahol azokat a felhasználók hozzáadták a térképhez. Előnyére szolgál, hogy a változtatások (pl. ha egy új bicikli utat szeretnénk hozzáadni), 24 órán belül bekerülnek az adatbázisba, és egy-két napon belül  láthatóvá válik mindenki számára (a \ref{fig:FELH:kep1}  kép, egy általunk hozzáadott bicikliutat és szerkesztési folyamatát ábrázolják).


\begin{figure}[t]
  \centering
  \begin{tabular}{ccc}
		  \pgfimage[height=5cm]{images/cycleMapEdit}
		  &
		  \pgfimage[height=5cm]{images/cycleMap}
	\end{tabular}
  \caption[Egy általunk hozzáadott bicikliút és szerkesztési folyamata]%
  {A jobb oldalon egy általunk hozzáadott bicikliút és a bal oldalon pedig a szerkesztési folyamata látható.\\
  {\white .}\url{}}
  \label{fig:FELH:kep1}
\end{figure}





%




\appendix
%%!TEX root = dolgozat.tex
%%%%%%%%%%%%%%%%%%%%%%%%%%%%%%%%%%%%%%%%%%%%%%%%%%%%%%%%%%%%
\chapter{Fontosabb programkódok listája}\label{ch:progik}

%%%%%%%%%%%%%%%%%%%%%%%%%%%%%%%%%%%%%%%%%%%%%%%%%%%%%%%%%%%%%%%%%
\lstset{language=Prolog}

Itt van valamennyi Prolog kód, megfelelõen magyarázva (komment-elve). A programok beszúrása az\\
\verb+\lstinputlisting[multicols=2]{progfiles/lolepes.pl}+\\
paranccsal történik, és látjuk, hogy a példában a \verb+progfiles+ könyvtárba tettük a file-okat.

Az alábbi kód Prolog nyelvbõl példa. Az \verb+\lstset{language=Prolog}+ paranccsal a programnyelvet változtathatjuk meg, ezt a \code{listings} csomag teszi lehetõvé \cite{listingCite}, amely nagyon jól dokumentált.

\lstinputlisting[multicols=2]{progfiles/lolepes.pl}




{ \renewcommand{\baselinestretch}{0.8}\normalsize %
  \setlength{\itemsep}{-2.4mm}
  \setlength{\bibspacing}{0.67\baselineskip}
  \bibliographystyle{abbrvnat_hu}
  \bibliography{dolgozat}
}

\end{document}
